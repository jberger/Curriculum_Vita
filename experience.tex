\begin{rubric}{Professional Experience}

\subrubric{Professional}

  \entry*[Jun 2013 --- \ldots] Python developer, forex trade entry and risk management platform, Bank of America Corporation

\subrubric{Computing}

  \entry*[Proficient in] Perl, Python, C/XS, Mathematica, Javascript, \LaTeX{} (Beamer/TikZ), Linux, HTML, CSS, Gnuplot
  \entry*[Familiar with] PHP, Modelica, SQL, MongoDB, Agile development
  \entry*[Concepts] Numerical simulation, OOP, testing, web frameworks, CMS, ORM, DOM, VCS (Git/GitHub), asynchronous/evented programming, websockets
  \entry*[Code] \url{http://search.cpan.org/~jberger}, \url{https://github.com/jberger}
  \entry*[Affiliations] \texttt{Mojolicious} (web framework) core developer, Recipient of a Perl Foundation grant, Perl Data Language (\texttt{PDL}) core developer (PDL porters), Chicago Perl Mongers
  %\entry*[Perl] Numerical simulation and analysis (native and PDL), c.f. \texttt{Physics::UEMColumn}
  
\subrubric{Research}

  \entry*[2005 --- 2013] UIC Ultrafast Physics Group. Advisor: Prof. W. Andreas Schroeder. Ultrafast Elecrton Miroscopy.
  \entry* Developed efficient model with high-level API for ultrafast electron pulse simulation. 
  \entry* Designed and constructed prototype UEM column.
  \entry* Performed experimental and theoretical work toward selecting an optimum photocathode material.
  \entry*[2003 --- 2005] UIC Microphysics Laboratory. Advisors: Prof. Sivalingham Sivananthan, Dr. Yong Chang.
  \entry* Minority Carrier Lifetime measurement and experiment maintenance and improvement.

\subrubric{Teaching}

  \entry*[Teaching] Taught: Problem-Solving Workshop for General Physics II (Electricity and Magnetism) (PHYS145 - Fall 2012, Spring 2013) 
  \entry*[Teaching] Co-Taught: \LaTeX{} for Technical Publishing (PHYS491 - Spring 2011)
  \entry*[T.A.] Laboratory Instructor: General Physics II (Electricity and Magnetism) (PHYS142 - Fall 2012)
  \entry*[T.A.] PHYS106 Mechanics discussion, PHYS402 E\&M grader

\end{rubric}
